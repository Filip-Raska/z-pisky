\documentclass{article}
\usepackage[utf8]{inputenc}
\usepackage{hyperref}
\usepackage{multirow}
\usepackage{amsmath}
\usepackage{pgfplots}

\title{FYZ zápisky}
\author{xrasfi01 }
\date{December 2021}

\begin{document}

\maketitle

\section*{O obecniny}
\subsection*{O1 o fyzice}
\begin{footnotesize}
"Fyzika je jedním z vrcholných stvoření lidského intelektu. Přípomínajíc svým rozmachem do výše se vypínající knleby středověkých katedrál, a přece nesouc tolik i z křehce nastíněné elegance veklých malířů, sochařů a skladatelů, ztělesňuje fyzika uměleckou tvořivost nejvyššího řádu. K této kráse estetické přidejme intelektuální triumfy, které tvarovaly naši histroii a inspirovaly naše sny, a též odvážného i obětavého ducha, jenž navenek dosáhl ke hvězdám a uvnitř vstoupil až do skrytých pevností atomového jádra. S počátky ve světě pověry a strachu z neznáma je příběh fyziky historií duchovního, uměleckého a intelektuálního výkonu, jaký v lidském dobrodružství nemá obdoby." 
\end{footnotesize}
\\
\\
\textbf{Dnes většinou:}
\\
- technické vymoženosti, ulehčení práce - sporné (neztrácíme smysl bytí?) \\
- samo slovo pochází z řeckého fýzis = příroda (nauka o přírodě, přírodních jevech) = přirozenost - přirozenost vnitřní podstata vede k archetypům (z vyššího světa)
\\ 
\\
\textbf{Metody:}
\\
Tradiční dělení fyziky (vědy obecně) na teoretickou a experimentální doplnila po válce počítačova technika a mohutný rozvoj počítačů znamená zpřístupnění širokého pole dosud neřešitelných problémů. Fyzika tak aspiruje přiblížit se tradičnímu ideálu pansofie čerpajíc ze 3 nezávislých zrojů poznání \href{(https://en.wikipedia.org/wiki/Immanuel_Kant)}{(Kant)}\\
\emph{- pozorování}\\
\emph{- experiment }\\
\emph{- intuice duševní (duchovní) inspirace}\\
Usilujeme o to, aby si dílčí poznatky z těchto sfér neodporovaly. \\
\\
\begin{footnotesize}
"... jednou z nejmocnějších složek umění je fantazie, ta, která přeměňuje suchý poznatek vjemový a rozumový a citový v kouzelné dílo básnické, ta, která stvořila Metamorfózy Ovidiovy a všech skutečných básnických duchů. Vjem, rozum, cit, idea, to jsou matérie každého člověka a z těch básník vytváří obrazotvorností novou, vyšší skutečnost. ... fantazie není chaosem a anarchií, nýbrž psychickou mocností, složkou, která třídí, formuje a dobývá. 
\end{footnotesize} \\
\\
Úspěchy poznání korigujme skromností.\\\\
\begin{huge}
KLEKÁNICE \\
Vladimír Holan
\end{huge}\\
\\
"Bříza v lese ošlehává podrost a dusí buřeň,"\\
řekl hajný.\\
"Ve vode je tomu stejně! \\
Tam okoun je taky plevelná ryba," \\
řekl baštýř. \\\\
Ještě že šla za chvíli kolem klekánice!\\
Oba zmlkli a všechno zase mohlo milovat\\
všechny stromy a všechny ryby\\
z pradávné známosti ...\\\\
\textbf{jazyk:}\\
Od Galilea (1546-1642) je jazykem fyziky matematika, dnes v podobě běžnému smrtelníkovi nedosažitelná. Potíže máme tehdy, zůstáváme-li v okruhu poznání Kristetolova (-334-322) pro nějž vědeckým bylo pečlivé sbírání dat, pozorování přírody. 

\subsection*{O2 pohyb - interakce - struktura}
Každému soucnu je v obecné podobě přirozený pohyb. Zároveň o sobě jsoucna "vědí" - interagují spolu. Výsledkem setkání těchto 2 dispozic je vznik strutkury.\\\\
\textbf{př:}\\
- elektron + proton = struktura atom \\
- obdobně u slunenčí soustavy - planety + Slunce = sluneční soustava \\\\
Tento proces je obecný najdeme jej v CHEM, BI, SOCI, PSY, jazyku

\subsection*{O3 fyzikální veličiny a jednotky}
Vystálení jednotné soustavy měr a vah, obecně jednotek, trvalo staletí a dodnes přetrvávají kuriozity.\\\\
\begin{footnotesize}
(ruský) \textbf{pud} = 16.38 kg, \textbf{rozotoč} (mite) = 3.24 mg, \textbf{sydharb} - jednotka objemu (kapacita přístavu v Sydney), (malajský) \textbf{perpisang} - doba potřebná k snědení banánu, (finská) \textbf{poronkusema} - vzdálenost (asi 7.5 až 10 km), kterou ujdou sobi, než se zastaví a močí
\end{footnotesize}\\\\

\subsubsection*{SI mezinárodní soustava}
\begin{tabular}{|l|c|l|c|}
\hline
Základní veličina & Značka veličiny & Základní jednotka & Značka jednotky \\
\hline
délka & l & metr & m \\
mhotnost & m & kilogram & kg \\
čas & t & sekunda & s \\
elektrický proud & I & ampér & A \\
termodynamická teplota & T & kelvin & K \\
látkové množství & N & mol & mol \\
svítivost & I & kandela & cd \\
\hline
\end{tabular}\\\\
Zápis každé hodnoty měřené veličiny musí obsahovat dvojí infromace: kolik?, čeho? \\\\

\subsubsection*{násobné jednotky a dílčí jednotky}
\begin{tabular}{|l|c|l|c|}
\hline
Předpona & Značka & Násobek & Mocnina deseti \\
\hline
tera- & T & 1 000 000 000 000 & $10^{12}$\\
giga & G & 1 000 000 000 & $10^{9}$ \\
mega & M & 1 000 000 & $10^{6}$ \\
kilo & k & 1 000 & $10^{3}$\\
mili & m & 0.001 & $10^{-3}$ \\
mikro & $\mu$ & 0.000 001 & $10^{-6}$\\
nano & n & 0.000 000 001 & $10^{-9}$ \\
piko & p & 0.000 000 000 001 & $10^{-12}$ \\
\hline
\end{tabular}

\subsubsection*{skaláry a vektory}
\emph{skaláry} = veličiny určené pouze svou velikostí \\
\emph{vektory} = 2 infromace velikost + směr (šipka nad písmenem)

\subsection*{O4 přesnost poznání}
Paradoxně jedním z jedůležitějších poznatků vědy je zjištění, že až na triviální případy nic neznáme přesně. Každé naše měření je zatíženo chybou a  s touto nepřeností se musíme naučit zacházet. Přístupy jsou v zásadě 2.
\subsubsection*{04.1 platné cifry}
Počet platných cifer, na něž smíme zapsat výsledek měření, je tím větší čím je měření přesnější. O platných cirfách: \\
\begin{enumerate}
    \item nuly před první ne nulou, nejsou PC \\
        l = 0. 001 km 1 PC \\
    \item nula mezi PC je platná \\
        t = 105 s = 3 PC \\
    \item nuly na konci jsou sporné \\
        140 l = 2 PC nebo 3 PC \\
\end{enumerate}
Ve fyzice má platná cifra význam poznání, proto $1 \neq 1.0, 1 m \neq 1.0 m$ \\\\
Zacházíme-li s více veličinami ($v = \frac{s}{t}$) pak výsledek píšeme jen na tolik PC, kolik jich je u nejméně přesného měření. 
\subsubsection*{O4.2 chyba měření}
x = měřená veličina \\
$x_m$ = naměřená hodnota \\
$x_s$ = správná hodnota \\\\
\textbf{absolutní chyba měření} \\
$\Delta x \equiv$ polovina nejmenšího dílku měřění \\
\begin{enumerate}
    \item absolutní chybu zapisujeme vždy na 1 PC
    \item naměřená veličina se převede do stejnéh řádu, jako chyba
    \item při početní operaci se $\Delta(x_1+x_2) = \Delta x_1 + \Delta x_2$
\end{enumerate}
\textbf{relativní chyba} \\
$\delta x = \frac{\Delta x }{x_m}$ \\
$\delta (x \cdot y) = \delta(\frac{x}{y}) = \delta x + \delta y$ \\
Ježto $x^n = x \cdot x \cdot x ... x$, platí $\delta (x^n) = \delta x \cdot n$ \\
Avšak $\sqrt{x} = x^{\frac{1}{2}}$ a proto $\delta \sqrt{x} = \frac{1}{2} \cdot \delta \cdot x$
Relativní chyba i na více PC.

\subsection*{05 univerzální konstanty}
\subsubsection*{rychlost světla c}
Ve vakuuu $c = 300 000 km/s = 2.999 \cdot 10^8$ m/s. \\\\
Zároveň i elektromagnetická konstanta. Limit daný speciální teorií relativity (SCR). Stejnou rychlostí se šíří i gravitační vlny v obecné teorii relativity (GRT). Jednotka vzdálenosti = světelný rok (ly), vzdálenost uražená c za rok ve vakuu. \\\\
ly = $3 \cdot 10^5 km \cdot 3.1536 \cdot 10^7 = 9.4608 \cdot 10^16 km$
\subsubsection*{gravitační konstanta G}
Gravitační síla je univerzální, ale velmi slabá. 
\[
G = 0.667 \cdot 10^{10} \quad \frac{m^3}{s^2kg}
\]
\[
F = G \cdot \frac{m_1m_2}{r^2}
\]
Konstanta se objeví v každém stavu s gravitací.
\subsubsection*{Boltzmanova konstanta $k_B$}
\[
k_B = 1.38 \cdot 10^{-23} \quad \frac{J}{K}
\]
Univerzální konstanta vystupujív  Clapeyronově stavové rovnici plynů
\[
pV = k_BNT
\]
Rovnice je svojí povahou (viz RA: [pV] = [$\frac{F}{S}\cdot m^3$] = J joule) energetická, tj. podíl $\frac{pV}{N}$ nese význam řádové hodnoty energie jedné částice termodinamického souboru, neboli e $\cong k_BT$. Tento vztah je obsahem tzv. ekvipartičního theorému. 

\subsection*{O6 rozměrová analýza}
Každá fyzikální situace vytváří jednak svojí povahou (gravitační), svými geometrickými rozměry (rozměry prostoru v němž se odehrává), velikostí číselných hodnot zúčastněných veličin svůj vlastní specifický rámec.\\\\ 
\emph{př.} \\
Míč letí (gravitace) dů brány (2.5, 7m) kopnut vyl rychlostí (12m/s). Tehdy až je konkrétní úloha sebesložitější lze z tohoto rámce alespoň na úrovni řádového odhadu nejen eliminovat chybná řešení, ale dokonce řádově odhadnout řešení správná. 

\subsection*{O7 zákony zachování}
\textbf{V uzavřené soustavě} se součet hmotností látek, které vstupují do reakce rovná součtu hmotnosti látek, které reakcí vznikají.\\ 
\textbf{V izolované soustavě} zůstává ceková mechanická energie těles coby součet jejich energií potenciálních a energií konstantních. Celková hybnost izolované soustavy se zachovává. \\\\
\textbf{Zákon zachování elektrického náboje}\\
V elektricky izolované soustavě těles může náboj přecházet z tělesa na těleso, mohou vznikat ionty (látka se tzv. ionizuje) a zas mohou zanikat (ionzy se tzv. rekombinují), ale úhrnný elektrický náboj je stálý. Elektrický náboj nelze vytvořit a nelze jej zničit. 

\section*{A pohyb}
\subsection*{A0 statika}
\subsubsection*{A0.1 klid}
- vztažné těleso \\
- pohyb je relativní vůči něčemu \\\\
\textbf{statika tuhého tělesa}
O podmínkách rovnováhy, stability tělesa nebo soustavy těles, o druzích rovnováhy, o těžišti tělesa o stablitě konstrukce. Těžiště tělesa = působiště tíhové síly, která na těleso působí. \\\\
\textbf{moment síly} = $M = r \cdot f \cdot sin \alpha [N \cdot m]$ \\\\
\textbf{podmínky statické rovnováhy}
Těleso je v rovnovážné polože pokud:
\begin{enumerate}
    \item je v klidu
    \item výslednice všech sil je rovna nule
    \item má nulový moment sil
\end{enumerate}
Existují 3 rovnovážné polohy:
\begin{enumerate}
    \item stálá (stabilní) - po vychýlení se vrátí 
    \item vratká (labilní) - výchylka uvede sílu která těleso z rovnovážné polohy vzdálí
    \item volná (indiferentní) - těleso po vychýlení zůstane ve stejné výšce (míč)
\end{enumerate}
\subsubsection*{A0.3 hydrostatika}
\textbf{Pascalův zákon} \\
\emph{B. Pascal 1632-1662} \\
Chování tekutin (l, g) bylo pospáno už v 17 st. Klíčovou veličinou popisu je tlak.
\[
p = \frac{F}{S} \quad [\frac{N}{m^2}] \quad pascal
\]
Pascal zjistil, že tlak se v tekutinách šíří všemi směry. Souvilost tzv. hydrostatického tlaku s výškou sloupce tekutiny prozkoumal tzv. Torricelli. \\\\
\textbf{Hydrostatický tlak} \\
Hledáme tlak tekutiny na dno. 
\[
p = \frac{F}{S} = \frac{mg}{S} = \frac{V \cdot g \cdot \rho}{S} = h \cdot g \cdot \rho
\]\\
\textbf{Archimédův zákon}\\
\emph{Archimedes 287-212 př.n.l}\\
Budeme zkoumat, jak je těleso ponořené do tekutiny nadlehčováno. Výsledná síla tzv. tlaková síla (buoyoncy) \\\\
h1 = výška od hladiny po vršek válce h2 = výška hladiny po spodek válce $\rho_0$ = hustota kapaliny
\[
p_1 = h_1\rho_0g \quad p_2 = h_2\rho_0g
\]
\[
F_1 = p_1S \quad F_2 = p_2S
\]
Tedy takto vzniká vzhůru působící tzv. vztlaková síla
\[
F_{vz} = F_2 - F_1 = Sg\rho_0 \cdot (h_2 - h_1) = Sg\rho_0h = \rho_0gV 
\]
$F_{vz} =$ tíze tekutiny tělesem vytlačené. To jest Archimédův zákon. \\\\
\textbf{srovnání s tíhou tělesa}\\
O ponoru tělesa tedy rozhodují hustoty těles a tekutiny:\\
\begin{align*}
    \rho > \rho_0 \quad \text{potopí} \quad se\\
    \rho = \rho_0 \quad \text{vznáší} \quad se\\
    \rho < \rho_0 \quad \text{plove} 
\end{align*}

\subsection*{A1 kinematika}
\subsubsection*{A1.1 pohyb rovnoměrný}
Nejjednoduší = klid. Rovnoměrná pohyb je příblížením k ostatním pohybům. \\\\
\emph{ú šnek}\\
1. naměřená data\\\\
\begin{tabular}{l|c|c|c|c}
    s (cm) & 2.0 & 4.0 & 6.0 & 8.0 \\
\hline
    t (min) & 2.5 & 5.2 & 7.4 & 10 
\end{tabular}\\\\
2. grafické znázornění \\\\
\begin{tikzpicture}
\begin{axis}[
    title={t~s},
    xlabel={s[cm]},
    ylabel={t[min]},
    xmin= 0, xmax=12,
    ymin= 0, ymax=12,
    xtick={0, 2, 4, 6, 8, 10, 12},
    ytick={0, 2, 4, 6, 8, 10, 12},
    legend pos= north west,
    ]
\addplot[
    color=blue,
    mark=square,
    ]
    coordinates{
    (0,0)(2,2.5)(4,5.2)(6,7.4)(8,10)};
\legend{t~s}

\end{axis}
\end{tikzpicture}\\\\
3. konstanta přímé úměrnosti\\
$t = p \cdot s$\\
$p = \frac{t_0}{s_0}$\\\\
4. shrnutí\\
Někdy konstanta pojmenovaná opačně než jsme si určili.
\begin{align*}
    k' = \frac{1}{k} = \frac{1}{p} = v = rychlost\\
    t = \frac{s}{v} \quad s = \frac{v}{t}
\end{align*}
v = rychlost rovnoměrného pohybu 1m/s = 3.6 km/h. V životě pohyb nebývá rovnoměrný, počítáme $v = \frac{s}{t}$ jako průměrnou rychlost.\\\\
V = první rychlost v = druhá rychlost s = vzdálenost
\begin{align*}
    v_{\text{průměrná}} \neq \frac{V+v}{2}\\
    v_{\text{průměrná}} = 2s \cdot (\frac{V}{s}+\frac{v}{s})\\
    v_{} = 2sVv
\end{align*}
Rychlost zvuku ve vzduchu = c = 340m/s (značíme \emph{c} pouze pokud nehrozí záměna s rychlostí světla)

\subsubsection*{A1.2 rovnoměrný otáčivý pohyb}
$\varphi$ = úhel\\
$\varphi$ ~ t\\
$\varphi = k \cdot t \quad$ $\varphi = w \cdot t$\\
$w = \varphi_0 / t_0 \quad$ k = w = úhlová rychlost\\\\
2 typy:\\
rotační\\
revoluční = oběžný\\

\subsection{A1.3 rovnoměrně zrychlený pohyb}
Brždění vozidla, omezíme se na rovnoměrné změny.\\\\
\textbf{1. data}\\\\
\begin{tabular}{|l|c|c|c|c|}
\hline
    t(s) & 0 & 1 & 2 & 3\\
\hline
    v(m/s) & 0 & 2.0 & 3.9 & 6.1\\
\hline
\end{tabular}\\\\
\textbf{2. graf}\\
\begin{tikzpicture}
\begin{axis}[
    title={v~t},
    xlabel={t[s]},
    ylabel={v[m/s]},
    xmin= 0, xmax=7,
    ymin= 0, ymax=7,
    xtick={0, 1, 2, 3, 4, 5, 6, 7},
    ytick={0, 1, 2, 3, 4, 5, 6, 7},
    legend pos= north west,
    ]
\addplot[
    color=blue,
    mark=square,
    ]
    coordinates{
    (0, 0)(1, 2)(2, 3.9)(3, 6.1)};
\legend{v~t}
\end{axis}
\end{tikzpicture}\\
\textbf{3. tedy $\exists$ k}\\
v~t\\
$v = k \cdot t$\\
K se jmenuje a (acceler = rychlý) zrychlení a.\\
$v = a \cdot t$\\\\
\textbf{Rovnoměrně zrychlený pohyb}
\begin{align*}
v(t) = v_0 + t \cdot a\\
s(t) = v_0 \cdot t + \frac{1}{2} \cdot a \cdot t^2
\end{align*}
\textbf{Rovnoměrně zpomalený pohyb}
\begin{align*}
    v(t) = v_0 - t \cdot a\\
    s(t) = v_0 \cdot t - \frac{1}{2} a \cdot t^2
\end{align*}
\begin{tikzpicture}
\begin{axis}[
    title={$s(t) = v_0 \cdot t + \frac{1}{2} \cdot a \cdot t^2$},
    xlabel={t},
    ylabel={s},
    ]
\addplot[
    domain=0:10,
    samples=100,
    color=blue,
    ]
    {1/2*10*x^2}
\addlegendentry{\ $\frac{1}{2} \cdot a \cdot t^2$\}
\end{axis}
\end{tikzpicture}

\section*{B interakce}
\section*{AB struktura}
\section*{D dodatek}


\end{document}















































